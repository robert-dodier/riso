\documentclass[12pt]{article}
\usepackage[letter,colorlinks]{hyperref}

\title{How to install RISO}
\author{Robert Dodier\\{\tt robert\_dodier@yahoo.com}\\{\tt http://riso.sourceforge.net}}
\date{\today}

\newcommand{\RISO}{{\sc riso}}
\begin{document}
\maketitle

\section{Overview}

\RISO is a Java implementation of heterogeneous distributed belief network algorithms.
\RISO allows one to load belief networks onto various hosts,
and link the individual belief networks into one grand belief network.
The polytree inference algorithm is employed, in which messages
in the form of probability distributions and likelihood functions
are passed between hosts; the communications medium is the Internet.
\RISO uses the Remote Method Invocation (RMI) system of Java
to handle message passing; this makes it possible to write code
in which messages look like function return values, but in reality
the messages are blocks of data being passed over the Internet.

\RISO was originally developed by the present author as part of
his doctoral research.
There is now a project page at SourceForge, namely {\tt http://riso.sourceforge.net},
with links to source and binary code, and additional documents.
\RISO may be copied, modified, and redistributed under the terms of the
GNU General Public License.

\section{System architecture}

The complete \RISO software is installed on one host,
<tt>civil.colorado.edu</tt>,
at the Dept. of Civil Engineering, University of Colorado.
The software, in the form of Java <tt>.class</tt> files,
is accessible to the outside world via the Web server on <tt>civil</tt>.
\RISO classes (i.e., blobs of compiled code) are requested automatically
by the Java run-time environment (JRE) on a remote host --- thus, only the JRE 
and a short stub application need to be installed on the remote host.
Any \RISO classes needed are downloaded to the remote host on demand,
when a \RISO application is started by the stub application.

\RISO uses Remote Method Invocation (RMI) to copy function arguments and
return values from one host to another.
RMI in turn is based on socket connections, and handles all the details of
managing connections for \RISO --- from a programmer's perspective, RMI is
very convenient.

A variable in a belief network may name any other variable, in any other
belief network on any other host, as a parent.
This opens a connection (managed by RMI) between the host containing the child
and the host containing the parent. Probability messages may flow in either
direction on this connection.
If the referred-to parent can't be located, an attempt is made to have 
the parent's belief network loaded onto the parent host;
this is similar in spirit to the resolution of names in ordinary function libraries.

A reference to a belief network can be obtained from the RMI registry,
maintained by the RMI software.
The registry maintains a list of belief network names and pointers to the
corresponding belief networks.
The registry listens for connections on port 1099 (by default), and 
translates names into references;
this is the mechanism used by child variables to obtain pointers to their parents.
When a new belief network is created,
its name is put into the registry.

Belief networks are created and maintained by software called the 
belief network context. 
The context also has a name which is published in the RMI registry --- if you
want to start a belief network on a remote machine, you start by finding a 
context on that machine and requesting the context to start the belief network.
In general, belief networks can be loaded from the local filesystem or described
by a string.
Since the stub application prohibits access to the local filesystem,
belief networks can only be started remotely by sending a description string
to a belief network context.
Descriptions are typically several hundred bytes to a few thousand bytes in length.

\section{Security}
At present (June, 2002), \RISO includes only those security measures
provided by Java. 
These measures can prevent some denial of service and spoofing attacks.
\RISO uses plain sockets, so all messages are transported in
the clear over the Internet.

The \RISO application stub (<tt>RemoteApp</tt>) executed on the
host machine prohibits any code which it loads from either reading
or writing the local filesystem.
The application stub allows unlimited socket connections --- these
connections are for the purpose of connecting to other belief network hosts.
The execution priority of the application stub (and so, the priority of
any belief network code) can be adjusted to suit the taste of the
administrator of the host, by the <tt>renice</tt> command, for example.

\section{Installation details}

\begin{enumerate}

It is assumed that there is a Java Runtime Environment (JRE) running on each \RISO host.
The code has been tested with IBM's implementation of JRE 1.3.0 for Linux;
\RISO should work equally well on other platforms.
Note that JRE versions 1.1.x use a different RMI implementation than 1.2.x and later.

\subsection{Isolated host or multiple independent hosts}

\item {\tt ???}
\item {\tt rmiregistry \&}

\item {\tt java riso.belief\_nets.BeliefNetworkContext -c} {\it context-name}

\subsection{Serving class files to dependent hosts}

\item Copy the stub application, <tt>RemoteApp</tt>. This comprises two class files:

	<ul>
	<li> <a href="http://civil.colorado.edu/~dodier/java/RemoteApp.class">
		RemoteApp.class</a>
	<li> <a href="http://civil.colorado.edu/~dodier/java/RemoteAppSecurityManager.class">
		RemoteAppSecurityManager.class</a>
	</ul>

	Just shift-click on these links to download the files.
	NOTE: Some browsers may rename the <tt>.class</tt> files to something else,
	such as <tt>.exe</tt>. Check the downloaded files to see the filename extension ---
	if it's not <tt>.class</tt>, rename the files.

	<p> Put the <tt>.class</tt> files in a directory named <tt>\RISO</tt>.
	You can browse all of <a href="http://civil.colorado.edu/~dodier/java">
	the compiled \RISO software</a> if you like.
	<p>

\item Execute the RMI registry. This program produces no output.
	<p>
	<ul>
	<li> Unix: put this in the background.
<pre>
rmiregistry &
</pre>
	<p>
	<li> Windows: double-click on <tt>rmiregistry.exe</tt>.
	The JRE install program puts the binaries in 
<pre>
\Program Files\JavaSoft\JRE\1.1\bin
</pre>
	unless you tell it otherwise.
	A window will open, 
		although no output will be display.
	</ul>
	<p>

\item Create a belief network context using the stub application.
	<p>
	<ul>
	<li> Unix:
	<p>
	Set environment variables to run the stub application.
	<ul>
	<li> <tt>setenv CLASSPATH RISO":"$CLASSPATH</tt>
	<li> <tt>alias jre 'jre -Djava.rmi.server.codebase=http://civil.colorado.edu/~dodier/java/'</tt>
	</ul>
	<p>
	IMPORTANT: Don't forget the trailing backslash on the codebase!
	<p>
	Now execute the stub and tell it to execute the context in turn.
<pre>
jre RemoteApp -c riso.belief_nets.BeliefNetworkContext -a -c my-context
</pre>
	You may want to redirect the debug output (of which there's plenty) of the
	belief network context to <tt>/dev/null</tt>.
	It would be very helpful, for the purposes of the \RISO demonstration,
	to put <tt>jre</tt> in a shell loop, so that if the belief network context
	dies, it will restart. Here is one way to arrange that:
<pre>
#!/bin/csh
while (1)
  jre RemoteApp -c riso.belief_nets.BeliefNetworkContext -a -c context-name
end
</pre>
	Adjust the execution priority of <tt>jre</tt> as needed, using <tt>renice</tt>,
	for example.
	<p>

	<li> Windows: Add the directory containing <tt>RemoteApp.class</tt> to the
	classpath, and specify the codebase.
<pre>
jre -cp RISO -Djava.rmi.server.codebase=http://civil.colorado.edu/~dodier/java/ RemoteApp -c riso.belief_nets.BeliefNetworkContext -a -c my-context
</pre>
	IMPORTANT: Don't forget the trailing backslash on the codebase!
	</ul>
	<p>

<li> <tt>RemoteApp</tt> can execute any \RISO application, if you are so inclined.
	See the <a href="bnc-pn-rq.html">notes
	on some \RISO apps,</a> and these
	<a href="riso-misc.html">
	other notes on \RISO apps.</a>
	The general syntax for <tt>RemoteApp</tt> is (on Unix; modify by adding <tt>-cp</tt> and
	<tt>-Djava.rmi.server.codebase</tt> for Windows)
<pre>
jre RemoteApp -c NameOf_RISO_ClassHere -a ArgsForAppHere
</pre>
	
	<p> For example,
<pre>
jre RemoteApp -c RegistryPing -a SomeHostName
</pre>
	tells what belief networks and contexts are running on a particular host.
</ol>

<hr>
<a href="index.html">\RISO home page.</a>
<p>
Last update sometime before <tt>$Date$</tt>.
</body>
